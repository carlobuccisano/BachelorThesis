\chapter*{Introduzione}
	Questa tesi si compone di due parti: nella prima (capitoli 1-3) si costruiscono i vari campi $p$-adici per arrivare a $\Cp$, campo completo e algebricamente chiuso, tentando di emulare il procedimento classico che da $(\Q, \abs{\ }_{\infty})$ porta a $\C$. Nella seconda parte (capitoli 4-5), invece, si studiano le funzioni analitiche su $\Cp$, definite come serie di potenze, con particolare attenzione ad alcune funzioni elementari, come l'esponenziale e il logaritmo, e alle differenze che si presentano nel caso $p$-adico rispetto al caso classico. Infine, nel capitolo 5, viene definito il poligono di Newton, potente strumento per capire subito il raggio di convergenza e l'ordine ($p$-adico) degli zeri di una funzione analitica. \newline
	Più specificamente, nel capitolo 1, dopo aver introdotto la definizione di norma e quella di valore assoluto $p$-adico su $\Q$ (e aver mostrato che definisce una norma non-Archimedea), si dimostra il teorema di Ostrowski, che afferma che ogni norma non banale su $\Q$ è equivalente o al valore assoluto classico o a un valore assoluto $p$-adico, per qualche primo $p$. Viene poi provato che $(\Q, \pabs{\ })$ non è completo e viene definito, nel modo classico, il suo completamento $(\Qp, \pabs{\ })$, analogo di $(\R, \abs{\ }_{\infty})$ nel caso classico. Infine, viene provato un teorema di struttura, che afferma che ogni elemento di $\Qp$ può essere scritto come una serie del tipo $\sum_{i=m}^{+\infty} a_ip^i$, con $a_i \in \{0, \dots, p-1\}$ e $m \in \Z$. \newline
	Nel capitolo 2 si arriva di nuovo a costruire $(\Qp, \pabs{\ })$ in un modo, però, totalmente diverso dal primo e più ``algebrico''. Si parte infatti da $\Zp$, insieme contenente tutti gli elementi del tipo $\sum_{i=0}^{+\infty} a_ip^i$ con $a_i \in \{0, \dots, p-1\}$, equipaggiato con le operazioni di somma con riporto e prodotto alla Cauchy (con riporto). Si mostra che esso è un dominio integrale (qui si capisce perché $p$ debba essere primo) e che contiene gli interi (o meglio, che esiste un monomorfismo $\Z \hookrightarrow \Zp$). Dopo un breve excursus su proprietà generiche dei gruppi topologici (dalle quali si ricaverà che $\Zp$ è uno spazio compatto, completo e metrizzabile), viene mostrata un'altra definizione di $\Zp$: come limite proiettivo degli insiemi $\Z/p^n\Z$. Infine si mostra che $\Qp$ è esattamente il campo delle frazioni di $\Zp$ e si introduce il lemma di Hensel, fondamentale strumento per ``rialzare'' le radici di polinomi da $\Z/p^n\Z$ a $\Zp$, quando il polinomio soddisfa opportune ipotesi. \newline
	Nel capitolo 3, dopo un breve excursus su generiche proprietà di spazi ultrametrici, si studiano le estensioni di campi $K/\Qp$ di grado finito e si vede come si può estendere il valore assoluto $p$-adico a tali campi $K$. Vengono poi classificate in base al loro indice di ramificazione e al grado residuo, con particolare attenzione ad estensioni non ramificate e totalmente ramificate. Dopo aver mostrato una versione analoga del criterio di Eisenstein nel caso $p$-adico, viene mostrato che $\Qp$ ammette estensioni di qualunque grado finito, da cui si ricava che la sua chiusura algebrica, $\Qpa$, ha necessariamente grado infinito su $\Qp$. Infine si mostra che $(\Qpa, \pabs{\ })$ non è completo e si considera il suo completamento $\Cp$, che si mostrerà essere anche algebricamente chiuso. Si noti che qui il caso $p$-adico sembra essere più complicato del caso classico: ciò è dovuto al fatto che $\Qpa$ ha grado infinito su $\Qp$ e dunque la completezza si ``perde'', mentre nel caso classico $\C = \R^{\textup{alg cl}}$ ha grado finito (2) su $\R$ e dunque rimane completo. Infine viene dimostrato un teorema di struttura di $\Cp$, che afferma che ogni elemento è prodotto di una potenza frazionaria (radice di un polinomio del tipo $X^a - p^b$, con $a,b \in \Z$), una radice di 1 e un elemento nel disco aperto di raggio 1 centrato in 1. In realtà, come spiegato alla fine del capitolo 3, i due processi (di costruzione di campi completi e algebricamente chiusi) possono essere fatti in modo totalmente analogo: si considera prima $\Q$, poi la sua chiusura algebrica $\Q^{\textup{alg cl}}$ e si completa quest'ultima rispetto a $\abs{\ }_{\infty}$ per ottenere $\C$ e rispetto a $\pabs{\ }$ per ottenere $\Cp$. Il problema di questa costruzione è la notevole difficoltà che si incontra nello studio di $\Q^{\textup{alg cl}}$.\newline
	Nel capitolo 4 viene introdotta la nozione di funzione analitica su $\Cp$, funzione definita come una serie di potenze (dove esssa converge). Viene provato poi che la stessa formula classica per trovare il raggio di convergenza di una serie di potenze vale anche nel caso $p$-adico (sostituendo chiaramente il valore assoluto $p$-adico a quello classico). Viene anche introdotta le definizione di differenziabilità (e stretta differenziabilità) e viene provato che le funzioni analitiche sono differenziabili nel modo standard (termine a termine). Vengono poi definite le funzioni $\exp_p(X)$ e $\log_p(1 + X)$ (usando le serie di MacLaurin note dal caso classico) e viene provato che, a differenza del caso reale, la funzione esponenziale converge solo su un piccolo disco aperto centrato in 0 (di raggio $r_p = p^{-1/(p-1)}$). Le proprietà classiche di esponenziale e logaritmo, però, si conservano anche nel caso $p$-adico e, restringendo in maniera appropriata dominio e codominio, si mostra che esponenziale e logaritmo sono funzioni l'una inversa dell'altra. Infine si introducono due nuove funzioni: il logaritmo di Iwasawa e l'esponenziale di Artin-Hasse. La prima è una funzione localmente analitica, definita su tutto $\Cp$, che estende il logaritmo precedentemente definito e ha derivata $x \mapsto 1/x$. L'esponenziale di Artin-Hasse, invece, è ricavato togliendo i termini ``problematici'' dall'esponenziale, ottenendo così una più estesa regione di convergenza (più specificamente viene prima mostrato un modo per scrivere $\exp_p(X)$ come prodotto infinito di serie di potenze e viene poi notato che sono solo alcuni di questi termini a imporre una minore regione di convergenza: togliendoli si ottiene l'esponenziale di Artin-Hasse). Nonostante il nome che potrebbe trarre in inganno, esso non è un'estensione dell'esponenziale: infatti vale $\E_p(X) = \exp_p\left(X + \tfrac{X^p}{p} + \tfrac{X^{p^2}}{p^2} + \dots\right)$. \newline
	Nell'ultimo capitolo viene introdotta la definizione di poligono di Newton prima per i polinomi e poi per le serie di potenze. Viene poi presentato l'importante teorema che lega gli zeri di un polinomio al suo poligono di Newton: infatti per ogni segmento di pendenza $\lambda$ e di lunghezza $M$ (qui per lunghezza si intende quella della proiezione sull'asse orizzontale) vi sono esattamente $M$ zeri, contati con molteplicità, di ordine $p$-adico $-\lambda$ e tutti gli zeri sono ottenuti in questo modo. Dopo aver mostrato che il raggio di convergenza di una serie è esattamente il $\sup$ delle pendenze del suo poligono di Newton si mostrano dei lemmi tecnici per arrivare a dimostrare il teorema di separazione di Weierstrass. Esso ha, tra i suoi corollari, la generalizzazione alle serie di potenze del teorema precedentemente enunciato solo per i polinomi, ossia per ogni segmento di lunghezza $N < +\infty$ e di pendenza $\lambda$ si hanno $N$ zeri della serie di ordine $p$-adico $-\lambda$. L'ultimo risultato mostrato è che ogni serie di potenze $f(X) \in 1 + X\Cp\ser{X}$ convergente su tutto $\Cp$ ha un insieme di zeri numerabile, sia $(r_n)_{n \in \N}$, e vale $f(X) = \prod_{n \in \N} \left(1 - \tfrac{X}{r_n}\right)$. Possiamo pensare a tale risultato come ad una generalizzazione del teorema fondamentale dell'algebra (esiste una versione di tale teorema anche su $\C$, ma si ottiene un prodotto con fattori più complicati). Tra le conseguenze di quest'ultimo vi è ad esempio il fatto che non può esistere, su $\Cp$, un esponenziale come nel caso classico, ossia ovunque convergente e mai nullo: infatti qualunque serie del genere deve essere una costante.